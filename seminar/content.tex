\documentclass[12pt, a4paper]{article}

\usepackage{ucs}
\usepackage[russian]{babel}
\usepackage{cmap}
\usepackage[utf8x]{inputenc}
\usepackage{listings}
\usepackage{float}
\usepackage[margin=20mm]{geometry}
\usepackage{url}
\usepackage{hyperref}

\frenchspacing

\thispagestyle{empty}

\begin{document}

\begin{center}
   \Large{Интеллектуальный анализ данных}
   \Large{(data mining)}
\end{center}

\section*{План презентации}

\begin{enumerate}
   \item Содержание
   \item Введение

   \item Типы задач, решаемые ИАД
      \begin{enumerate}
         \item \textbf{Классификация}
         \item \textbf{Кластеризация}
	 \item \textbf{Задача сокращения размерности}
         \item Прогнозирование
	 \item Регрессионный анализ
         \item Задача поиска ассоциативных правил
	 \item \ldots
      \end{enumerate}

   \item Этапы решения задачи
      \begin{itemize}
         \item Обучение с учителем
         \item Обучение без учителя
      \end{itemize}

   \item (Подробнее и с примерами применения) Классификация
   \item (аналогично) Кластеризация
   \item (аналогично) Сокращение размерности (метод главных компонент)

   \item Заключение
\end{enumerate}

\section*{Список литературы}

\begin{enumerate}
   \item Википедия (англ., рус.)
   \item \url{http://www.machinelearning.ru}
   \item \url{http://www.anderson.ucla.edu/faculty/jason.frand/teacher/technologies/palace/datamining.htm}
   \item \url{http://www.inftech.webservis.ru/it/database/datamining/ar2.html}
   \item Christopher D. Manning, Prabhakar Raghavan and Hinrich Schütze, \emph{Introduction to Information Retrieval}, Cambridge University Press. 2008
         \url{http://nlp.stanford.edu/IR-book/information-retrieval-book.html}
\end{enumerate}

\end{document}
